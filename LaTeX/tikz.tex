\usepackage{tikz, pgfplots}
\usetikzlibrary{calc}
\usetikzlibrary{math}
\usetikzlibrary{shapes}
\usetikzlibrary{patterns}
\usetikzlibrary{backgrounds}
\usetikzlibrary{positioning}
\usetikzlibrary{decorations.pathmorphing}

\usetikzlibrary{shapes.geometric}
\usetikzlibrary{positioning,fit,shapes.geometric,backgrounds, calc}
\usetikzlibrary{arrows,decorations.markings}
\usetikzlibrary{shapes.arrows}
\usetikzlibrary{patterns}
\tikzset{textnode/.style={inner sep=0pt,outer sep=0,execute at begin node={\strut}}}
\tikzstyle{state} = [textnode, circle, draw, inner sep=0pt, outer sep=0]

\usetikzlibrary{pgfplots.groupplots}
\usepgfplotslibrary{groupplots}
\usepgfplotslibrary{fillbetween}
                    
\pgfplotsset{every axis/.append style={
                    xlabel={$x$},          % default put x on x-axis
                    ylabel={$y$},          % default put y on y-axis
                    % label style={font=\sffamily\small},
                    % tick label style={font=\sffamily\small},
                    % xticklabel style = {font=\sffamily\small},
                    % title style = {font=\footnotesize\sffamily},
                    ylabel near ticks,
                    % y label style={font=\sffamily\small},
                    xlabel near ticks,
                    % x label style={font=\sffamily\small},
                    legend cell align={left},
                    % legend style={draw=none, font=\sffamily\normalsize},
                    },
                    legend image code/.code={
                    \draw[mark repeat=2,mark phase=2]
                        plot coordinates {
                        (0cm,0cm)
                        (0.15cm,0cm)        %% default is (0.3cm,0cm)
                        (0.3cm,0cm)         %% default is (0.6cm,0cm)
                        };%
                    }
                    }
\pgfplotsset{compat=newest}  

\tikzmath{%
    \figscale = 1.75;
    \nodesize = 0.06cm;
    \nodespacing = 1.5pt;
    \radbig = 360/5;
    \radtri = 360/3;
    \rottri = 360/6;
    \radpent = 360/5;
    \rotpent = 360/10;
    \arccurve = 0.5cm;
    \arcstraight = 0.0cm;
}

\tikzset{snake it/.style={%
    decoration={%
        snake, 
        amplitude = 0.2mm,
        segment length = 1mm,
        post length=1mm
    },
    decorate
}}
% \tikzcdset{arrow style=tikz, squiggle/.style={%
    % decoration={%
        % snake,
        % amplitude=.3mm,
        % segment length=2mm,
        % post length=1mm
    % },
    % rounded corners=.2pt,
    % decorate
% }}

% styles
\tikzstyle{blank} = [outer sep=8*\nodespacing, inner sep=\nodesize]
\tikzstyle{wavy} = [decorate,
                    decoration={%
                        snake,
                        amplitude=0.25mm,
                        segment length=2.7mm,
                        post length=0.0mm,
                        pre length=0.0mm
                    }]
\tikzstyle{cone} = [draw=pgrey, shorten <= -2pt]

\tikzstyle{add} = [pteal, draw opacity=1]
\tikzstyle{del} = [porange, draw opacity=1]

% nodes
\tikzstyle{node} = [very thick,
                    circle,
                    draw=pblack,
                    fill=pblack,
                    outer sep=\nodespacing,
                    inner sep=\nodesize,
                    text opacity=1]
\tikzstyle{add_node} = [node, add]
\tikzstyle{del_node} = [node, del]

% edges
\tikzstyle{edge} = [very thick, draw=pblack, opacity=1]
\tikzstyle{add_edge} = [edge, add, densely dotted]
\tikzstyle{del_edge} = [edge, del, densely dotted, wavy]

% text
\tikzstyle{text} = [draw=pblack,  % circle?
                    fill=pgrey,
                    outer sep=\nodespacing,
                    inner sep=\nodesize,
                    fill opacity=1,
                    draw opacity=1,
                    text opacity=1]
\tikzstyle{add_text} = [text, fill=pteal, fill opacity=0.25]
\tikzstyle{del_text} = [text, fill=porange, fill opacity=0.25]

\tikzstyle{nts} = [text,
                   rectancle
                   rounded corners=0.5mm,
                   very thick]
\tikzstyle{lhs} = [nts, outer sep=4*\nodespacing]
