% % new proof environment
% \expandafter\let\expandafter\oldproof\csname\string\proof\endcsname
% \let\oldendproof\endproof
% \renewenvironment{proof}[1][\proofname]{%
%     \vspace{-0.5\parskip}%
%     \oldproof[#1]
% }{%
%     \renewcommand{\qedsymbol}{\sc q.e.d.}
%     \qed
% }

% NEW PROOF ENVIRONMENT
\expandafter\let\expandafter\oldproof\csname\string\proof\endcsname
\let\oldendproof\endproof
\renewenvironment{proof}[1][\bfseries\proofname]{%
  \vspace{-2ex}%
  % \vspace{-0.75\baselineskip}%
  \oldproof[\bfseries #1]
}{\qed}

% STYLE FOR THEOREMS
\declaretheoremstyle[headfont=\bfseries\itshape,
                     bodyfont=\normalfont,
                     notefont=,
                     headpunct={.\hfill\break},
                     qed=\kanji{定理},  % 定理結束
                     % prefoothook={\hfill\break}
                     ]{theoremstyle}
\declaretheoremstyle[headfont=\bfseries\itshape,
                     bodyfont=\normalfont,
                     notefont=,
                     headpunct={.\hfill\break},
                     qed=\kanji{推論},  % 推論結束
                     % prefoothook={\hfill\break}
                     ]{corollarystyle}
\declaretheoremstyle[headfont=\bfseries\itshape,
                     bodyfont=\normalfont,
                     notefont=,
                     headpunct={.\hfill\break},
                     qed=\kanji{引理},  % 引理結束
                     % prefoothook={\hfill\break}
                     ]{lemmastyle}
\declaretheoremstyle[headfont=\bfseries\itshape,
                     bodyfont=\normalfont,
                     notefont=,
                     headpunct={.\hfill\break},
                     qed=\kanji{定義},  % 定義結束
                     % prefoothook={\hfill\break}
                     ]{definitionstyle}
\declaretheoremstyle[headfont=\bfseries\itshape,
                     bodyfont=\normalfont,
                     notefont=,
                     headpunct={.\hfill\break},
                     qed=\kanji{公理},  % 公理結束
                     % prefoothook={\hfill\break}
                     ]{axiomstyle}
\declaretheoremstyle[headfont=\bfseries\itshape,
                     bodyfont=\normalfont,
                     notefont=\bfseries\itshape,
                     headpunct={.\hfill\break},
                     qed=\kanji{範例},  % 範例結束
                     % prefoothook={\hfill\break}
                     ]{examplestyle}
\declaretheoremstyle[headfont=\bfseries\itshape,
                     bodyfont=\normalfont,
                     notefont=\bfseries\itshape,
                     headpunct={.\hfill\break},
                     qed=\kanji{演算法},  % 演算法結束
                     % prefoothook={\hfill\break}
                     ]{algorithmstyle}
\declaretheoremstyle[headfont=\bfseries\itshape,
                     bodyfont=\normalfont,
                     notefont=\bfseries\itshape,
                     qed=\kanji{直覺},  % 直覺結束
                     % prefoothook={\hfill\break}
                     ]{ideastyle}

\ifbook
  \declaretheorem[numbered=yes, style=theoremstyle, within=chapter]{theorem}
  \declaretheorem[numbered=yes, style=corollarystyle, within=chapter]{corollary}
  \declaretheorem[numbered=yes, style=lemmastyle, within=chapter]{lemma}
  \declaretheorem[numbered=yes, style=definitionstyle, within=chapter]{definition}
  \declaretheorem[numbered=yes, style=examplestyle, within=chapter]{exercise, example, counterexample}
  \declaretheorem[numbered=yes, style=algorithmstyle, within=chapter]{algorithm}
  \declaretheorem[numbered=yes, style=ideastyle, within=chapter]{idea}
\else
  \declaretheorem[numbered=yes, style=theoremstyle]{theorem}
  \declaretheorem[numbered=yes, style=corollarystyle]{corollary}
  \declaretheorem[numbered=yes, style=lemmastyle]{lemma}
  \declaretheorem[numbered=yes, style=definitionstyle]{definition}
  \declaretheorem[numbered=yes, style=examplestyle]{exercise, example, counterexample}
  \declaretheorem[numbered=yes, style=algorithmsylte]{algorithm}
  \declaretheorem[numbered=yes, style=ideastyle]{idea}
\fi
\declaretheorem[numbered=yes, style=axiomstyle]{axiom}

\declaretheorem[numbered=no, style=theoremstyle]{Theorem}
\declaretheorem[numbered=no, style=corollarystyle]{Corollary}
\declaretheorem[numbered=no, style=lemmastyle]{Lemma}
\declaretheorem[numbered=no, style=definitionstyle]{Definition}
\declaretheorem[numbered=no, style=axiomstyle]{Axiom}
\declaretheorem[numbered=no, style=examplestyle]{Exercise, Example, Counterexample}
\declaretheorem[numbered=no, style=algorithmsylte]{Algorithm}
\declaretheorem[numbered=no, style=ideastyle]{Idea}

\crefname{figure}{\emph{figure}}{\emph{Figure}}
\crefname{table}{\emph{table}}{\emph{Table}}

\crefname{theorem}{\emph{theorem}}{\emph{Theorem}}
\crefname{corollary}{\emph{corollary}}{\emph{Corollary}}
\crefname{lemma}{\emph{lemma}}{\emph{Lemma}}
\crefname{definition}{\emph{definition}}{\emph{Definition}}
\crefname{axiom}{\emph{axiom}}{\emph{Axiom}}
\crefname{exercise}{exercise}{Exercise}
\crefname{example}{example}{Example}

\newenvironment{case}[1][Case]
  % {\par\textit{#1:}\hfill\break}
  {\quote\textit{#1:}}
  % {\par\begin{mdframed}[backgroundcolor=background,
      % linecolor=background,
      % % hidealllines,
      % fontcolor=foreground,
      % roundcorner=5pt]%
  % \begin{mdframed}[backgroundcolor=foreground, fontcolor=background, roundcorner=5pt]
    % \textit{#1:}}
  % {\end{mdframed}
  {\endquote}

\newenvironment{code}[1][Code]
  {\textbf{#1:}\quote}
  {\endquote}

\ifalgorithms
  \newcounter{nalg}[chapter]
  \renewcommand{\thenalg}{\thechapter.\arabic{nalg}}
  \DeclareCaptionLabelFormat{algocaption}{\it Algorithm \thenalg}

  \lstnewenvironment{algorithm}[1][]
  {
    \refstepcounter{nalg}
    \captionsetup{labelformat=algocaption,labelsep=colon}
    \lstset{mathescape=true,
            frame=tB,
            numbers=left,
            numberstyle=\tiny,
            basicstyle=\scriptsize,
            keywordstyle=\color{black}\bfseries\em,
            keywords={,input, output, return, datatype, function, in, if, else, elif, for, foreach, while, not, begin, end, true, false, null, break, continue, let, and, or, }
            numbers=left,
            xleftmargin=.04\textwidth,
            #1}
  }
  {}
\else
\fi
