% BEGIN COLORS
    \definecolor{gruvred}{HTML}{cc214d}
    \definecolor{gruvorange}{HTML}{d65d0e}
    \definecolor{gruvaqua}{HTML}{689d6a}
    \definecolor{gruvgreen}{HTML}{98971a}
    \definecolor{gruvblue}{HTML}{458588}
    \definecolor{gruvpurple}{HTML}{b16286}

    \definecolor{kombu}{HTML}{004d40}  % dark green
    \definecolor{grey}{HTML}{bbbbbb}
    \definecolor{black}{HTML}{252422}

    \definecolor{colornd}{HTML}{0C2340}
    \definecolor{colorfiu}{HTML}{081E3F}
    \definecolor{colorfsu}{HTML}{782F40}
    \definecolor{colorchicago}{HTML}{800000}

    \definecolor{blue}{HTML}{1e88e5}
    \definecolor{green}{HTML}{4eb981}
    \definecolor{yellow}{HTML}{ffc107}
    \definecolor{orange}{HTML}{ff5430}  % ff5733, f0746e, f28522, ee7733, d65d0e
    \definecolor{red}{HTML}{d81b60}

    \definecolor{ablue}{HTML}{089099}
    \definecolor{agreen}{HTML}{6ac495}
    \definecolor{ayellow}{HTML}{fbce6b}
    \definecolor{aorange}{HTML}{f0746e}
    \definecolor{ared}{HTML}{dc3977}
    \definecolor{apurple}{HTML}{912282}

    \newcommand{\tableblue}[1]{\cellcolor{ablue}{\white{#1}}}
    \newcommand{\tablegreen}[1]{\cellcolor{agreen}{\white{#1}}}
    \newcommand{\tableyellow}[1]{\cellcolor{ayellow}{\white{#1}}}
    \newcommand{\tableorange}[1]{\cellcolor{aorange}{\white{#1}}}
    \newcommand{\tablered}[1]{\cellcolor{ared}{\white{#1}}}
    \newcommand{\tablepurple}[1]{\cellcolor{apurple}{\white{#1}}}

    \newcommand{\white}[1]{\textcolor{white}{#1}}
    \newcommand{\grey}[1]{\textcolor{grey}{#1}}
    \newcommand{\black}[1]{\textcolor{black}{#1}}
    \newcommand{\blue}[1]{\textcolor{blue}{#1}}
    \newcommand{\green}[1]{\textcolor{green}{#1}}
    \newcommand{\yellow}[1]{\textcolor{yellow}{#1}}
    \newcommand{\orange}[1]{\textcolor{orange}{#1}}
    \newcommand{\red}[1]{\textcolor{red}{#1}}

    \newcommand{\mwhite}[1]{\color{white} #1 \color{black}}
    \newcommand{\mgrey}[1]{\color{grey} #1 \color{black}}
    \newcommand{\mblack}[1]{\color{black} #1 \color{black}}
    \newcommand{\mblue}[1]{\color{blue} #1 \color{black}}
    \newcommand{\mgreen}[1]{\color{green} #1 \color{black}}
    \newcommand{\myellow}[1]{\color{yellow} #1 \color{black}}
    \newcommand{\morange}[1]{\color{orange} #1 \color{black}}
    \newcommand{\mred}[1]{\color{red} #1 \color{black}}

    \newcommand{\ewhite}[1]{\emph{\textcolor{white}{#1}}}
    \newcommand{\egrey}[1]{\emph{\textcolor{grey}{#1}}}
    \newcommand{\eblack}[1]{\emph{\textcolor{black}{#1}}}
    \newcommand{\eblue}[1]{\emph{\textcolor{blue}{#1}}}
    \newcommand{\egreen}[1]{\emph{\textcolor{green}{#1}}}
    \newcommand{\eyellow}[1]{\emph{\textcolor{yellow}{#1}}}
    \newcommand{\eorange}[1]{\emph{\textcolor{orange}{#1}}}
    \newcommand{\ered}[1]{\emph{\textcolor{red}{#1}}}

    \newcommand{\iwhite}[1]{\textcolor{white}{\itshape #1}}
    \newcommand{\igrey}[1]{\textcolor{grey}{\itshape #1}}
    \newcommand{\iblack}[1]{\textcolor{black}{\itshape #1}}
    \newcommand{\iblue}[1]{\textcolor{blue}{\itshape #1}}
    \newcommand{\igreen}[1]{\textcolor{green}{\itshape #1}}
    \newcommand{\iyellow}[1]{\textcolor{yellow}{\itshape #1}}
    \newcommand{\iorange}[1]{\textcolor{orange}{\itshape #1}}
    \newcommand{\ired}[1]{\textcolor{red}{\itshape #1}}

    \newcommand{\bwhite}[1]{\textcolor{white}{\bfseries #1}}
    \newcommand{\bgrey}[1]{\textcolor{grey}{\bfseries #1}}
    \newcommand{\bblack}[1]{\textcolor{black}{\bfseries #1}}
    \newcommand{\bblue}[1]{\textcolor{blue}{\bfseries #1}}
    \newcommand{\bgreen}[1]{\textcolor{green}{\bfseries #1}}
    \newcommand{\byellow}[1]{\textcolor{yellow}{\bfseries #1}}
    \newcommand{\borange}[1]{\textcolor{orange}{\bfseries #1}}
    \newcommand{\bred}[1]{\textcolor{red}{\bfseries #1}}
% END COLORS

\newcommand{\blankpage}{\newpage\hbox{}\thispagestyle{empty}\newpage}
\newcommand*{\xline}[1][3em]{\rule[0.5ex]{#1}{0.55pt}}

\newcommand{\isomorphic}{\cong}
\newcommand{\defniff}{~\mathrel{\vcentcolon\Leftrightarrow}~}
\newcommand{\iffdefn}{~\mathrel{\vcentcolon\Leftrightarrow}~}
\newcommand{\iffbydefn}{\(\mathrel{\vcentcolon\Leftrightarrow}\)\ }
\newcommand{\niff}{\mathrel{{\ooalign{\hidewidth$\not\phantom{"}$\hidewidth\cr$\iff$}}}}
\newcommand{\nimplies}{~\not\Rightarrow~}
\renewcommand{\implies}{~\Rightarrow~}
\renewcommand{\impliedby}{~\Leftarrow~}
\renewcommand{\iff}{~\Leftrightarrow~}
\newcommand{\proves}{~\vdash~}
\newcommand{\satisfies}{~\models~}
\renewcommand{\qedsymbol}{\sc q.e.d.}

\renewcommand{\restriction}[1]{\downharpoonright_{#1}}
\renewcommand{\leq}{\leqslant}
\renewcommand{\geq}{\geqslant}

\newcommand{\meet}{\wedge}
\newcommand{\join}{\vee}
\newcommand{\conjunct}{\wedge}
\newcommand{\disjunct}{\vee}
\newcommand{\bigmeet}{\bigwedge}
\newcommand{\bigjoin}{\bigvee}
\newcommand{\bigconjunct}{\bigwedge}
\newcommand{\bigdisjunct}{\bigvee}
\newcommand{\defn}{\coloneqq}
\newcommand{\xor}{\oplus}
\newcommand{\nand}{\uparrow}
\newcommand{\nor}{\downarrow}
\newcommand{\Plus}{\text{\begin{tikzpicture}[baseline=-0.56ex, x=1ex, y=1ex]
    \node at (-26/40, 0) {};
    \node at (26/40, 0) {};
    \draw [line width=1.25, line cap=round] (0, -2/3) -- (0, 2/3);
    \draw [line width=1.25, line cap=round] (-2/3, 0) -- (2/3, 0);
\end{tikzpicture}}}


\newcommand{\compose}{\circ}
\renewcommand{\div}{\mid}
\newcommand{\ndiv}{\nmid}
\newcommand{\divides}{\mid}
\newcommand{\notdivides}{\nmid}
\newcommand{\given}{~\middle|~}
\newcommand{\suchthat}{~\middle|~}

\iftufte
    \newcommand{\contradiction}{\smash{\text{\Large \Lightning}}~~}
\else
    \newcommand{\contradiction}{\smash{\text{\raisebox{-0.6ex}{\Large \Lightning}}}~~}
\fi

\newcommand{\conjugate}[1]{\overline{#1}}
\newcommand{\mean}[1]{\overline{#1}}

\newcommand*\diff{\mathop{}\!\mathrm{d}}
\newcommand{\integral}[1]{\smashoperator{\int_{#1}}}
\newcommand{\E}[1]{\mathbb{E}\sq*{#1}}
\newcommand{\Esub}[2]{\mathbb{E}_{#1}\sq*{#2}}
\newcommand{\var}[1]{\mathrm{Var}\pn*{#1}}
\newcommand{\cov}[2]{\mathrm{Cov}\pn*{#1, #2}}
\newcommand{\der}[2]{\frac{\diff{#1}}{\diff{#2}}}
\newcommand{\dern}[3]{\frac{\diff^{#3}{#1}}{\diff{#2}^{#3}}}
\newcommand{\derm}[3]{\frac{\diff^{#3}{#1}}{\diff{#2}}}
\newcommand{\prt}[2]{\frac{\partial{#1}}{\partial{#2}}}
\newcommand{\prtn}[3]{\frac{\partial^{#3}{#1}}{\partial{#2}^{#3}}}
\newcommand{\prtm}[3]{\frac{\partial^{#3}{#1}}{\partial{#2}}}
\newcommand{\modulo}[3]{#1 \equiv #2 ~\pn*{\mathrm{mod}~#3}}
\newcommand{\congruent}[3]{#1 \equiv #2 ~\pn*{\mathrm{mod}~#3}}
\newcommand*\id{\text{id}}

\newcommand{\inj}{\hookrightarrow}
\newcommand{\injection}{\hookrightarrow}

\newcommand{\surj}{\twoheadrightarrow}
\newcommand{\surjection}{\twoheadrightarrow}

\newcommand{\bij}{\lhook\joinrel\twoheadrightarrow}
\newcommand{\bijection}{\lhook\joinrel\twoheadrightarrow}

\newcommand{\monic}{\hookrightarrow}
\newcommand{\monomorphism}{\hookrightarrow}

\newcommand{\epic}{\twoheadrightarrow}
\newcommand{\epimorphism}{\twoheadrightarrow}

\newcommand{\iso}{\lhook\joinrel\twoheadrightarrow}
\newcommand{\isomorphism}{\lhook\joinrel\twoheadrightarrow}
\newcommand{\immersion}{\looprightarrow}

\renewcommand{\O}[1]{\mathcal{O}\pn*{#1}}
\renewcommand{\P}[1]{\mathbb{P}\pn*{#1}}
\newcommand{\power}[1]{\mathcal{P}\pn*{#1}}
\newcommand{\successor}[1]{\mathcal{S}\pn*{#1}}
\newcommand{\C}{\mathbb{C}}
\newcommand{\N}{\mathbb{N}}
\newcommand{\Q}{\mathbb{Q}}
\newcommand{\R}{\mathbb{R}}
\newcommand{\Z}{\mathbb{Z}}

% these don't need {} after them since they should be followed by text
\newcommand{\cf}{\textit{cf.},\ }
\newcommand{\eg}{\textit{e.g.},\ }
\newcommand{\ie}{\textit{i.e.},\ }
\newcommand{\etc}{\textit{etc.}\ }
\newcommand{\aka}{\textit{a.k.a.}\ }
\newcommand{\sic}{\textit{sic.}\ }
\newcommand{\viz}{\textit{viz.}\ }
\newcommand{\vide}{\textit{v.}\ }
\newcommand{\qv}{\textit{q.v.}\ }
\newcommand{\vv}{\textit{vice versa}}
\newcommand{\ifandonlyif}{\textit{iff}\ }
\newcommand{\textiff}{\textit{iff}\ }

% these need {} after them
\newcommand{\etal}{\textit{et al.}}
\newcommand{\wff}{\textit{wff}}

\DeclareMathOperator{\lcm}{lcm}
\DeclareMathOperator*{\argmin}{arg\!\min}
\DeclareMathOperator*{\argmax}{arg\!\max}

\let\originalleft\left
\let\originalright\right
\renewcommand{\left}{\mathopen{}\mathclose\bgroup\originalleft}
\renewcommand{\right}{\aftergroup\egroup\originalright}

\newcommand{\zh}[1]{\begin{CJK}{UTF8}{min}#1\end{CJK}}
\newcommand{\hanzi}[1]{\begin{CJK}{UTF8}{gkai}#1\end{CJK}}
\newcommand{\kanji}[1]{\begin{CJK}{UTF8}{bkai}#1\end{CJK}}
\newcommand{\kana}[1]{\begin{CJK}{UTF8}{min}#1\end{CJK}}

% parentheses and delimiters
\DeclarePairedDelimiterX \inner[2]{\langle}{\rangle}{#1,#2}
\DeclarePairedDelimiter \bra{\langle}{\rvert}
\DeclarePairedDelimiter \ket{\lvert}{\rangle}
\DeclarePairedDelimiter \abs{\lvert}{\rvert}
\DeclarePairedDelimiter \cardinality{\lvert}{\rvert}
\DeclarePairedDelimiter \order{\lvert}{\rvert}
\DeclarePairedDelimiter \norm{\lVert}{\rVert}
\DeclarePairedDelimiter \set{\lbrace}{\rbrace}
\DeclarePairedDelimiter \seq{\langle}{\rangle}
\DeclarePairedDelimiter \pn{(}{)}
\DeclarePairedDelimiter \sq{[}{]}
\DeclarePairedDelimiter \curly{\lbrace}{\rbrace}
\DeclarePairedDelimiter \bracket{\langle}{\rangle}

\let\oldemptyset\emptyset
\let\emptyset\varnothing
\let\union\cup
\let\Union\bigcup
\let\intersection\cap
\let\intersect\cap
\let\Intersection\bigcap
\let\Intersect\bigcap
