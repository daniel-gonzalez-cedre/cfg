% % new proof environment
% \expandafter\let\expandafter\oldproof\csname\string\proof\endcsname
% \let\oldendproof\endproof
% \renewenvironment{proof}[1][\proofname]{%
%     \vspace{-0.5\parskip}%
%     \oldproof[#1]
% }{%
%     \renewcommand{\qedsymbol}{\sc q.e.d.}
%     \qed
% }

% NEW PROOF ENVIRONMENT
\expandafter\let\expandafter\oldproof\csname\string\proof\endcsname
\let\oldendproof\endproof
\renewenvironment{proof}[1][\bfseries\proofname]{%
    \vspace{-0.5\baselineskip}%
    \oldproof[\bfseries #1]
}{%
    ~\\\qed
}

% STYLE FOR THEOREMS
\declaretheoremstyle[
    headfont=\bfseries\itshape,
    bodyfont=\normalfont,
    notefont=,
    % headpunct={.\hfill\shifttext{2ex}{$\urcorner$}\break},
    headpunct={.\hfill\break},
    % headformat={\makebox[0pt][r]{\NAME\ \NUMBER}\NOTE},
    % headformat={\llap{\NAME} \NUMBER \NOTE},
    % qed=\raisebox{-2ex}{\shifttext{2ex}{$\lrcorner$}},
]{theoremstyle}
\ifbook
    \declaretheorem[
        numbered=yes,
        style=theoremstyle,
        within=chapter,
    ]{theorem, corollary, proposition, lemma, claim}
\else
    \declaretheorem[
        numbered=yes,
        style=theoremstyle,
    ]{theorem, corollary, proposition, lemma, claim}
\fi
\declaretheorem[
    numbered=no,
    style=theoremstyle,
]{Theorem, Corollary, Proposition, Lemma, Claim}

% STYLE FOR AXIOMS
\declaretheoremstyle[
    headfont=\bfseries\itshape,
    bodyfont=\normalfont,
    notefont=,
    % headpunct={.\hfill\shifttext{2ex}{$\urcorner$}\break},
    headpunct={.\hfill\break},
    % headformat={\llap{\NAME} \NUMBER \NOTE},
    % qed=\raisebox{-2ex}{\shifttext{2ex}{$\lrcorner$}},
]{axiomstyle}
\declaretheorem[
    numbered=yes,
    style=axiomstyle,
    % within=chapter,
]{axiom}
\declaretheorem[
    numbered=no,
    style=axiomstyle,
]{Axiom}

% STYLE FOR DEFINITIONS
\declaretheoremstyle[
    headfont=\bfseries\itshape,
    bodyfont=\normalfont,
    notefont=,
    % headpunct={.\hfill\shifttext{2ex}{$\urcorner$}\break},
    headpunct={.\hfill\break},
    % headformat={\llap{\NAME} \NUMBER \NOTE},
    % qed=\raisebox{-2ex}{\shifttext{2ex}{$\lrcorner$}},
]{definitionstyle}
\ifbook
    \declaretheorem[
        numbered=yes,
        style=definitionstyle,
        within=chapter,
    ]{definition, notation, algorithm}
\else
    \declaretheorem[
        numbered=yes,
        style=definitionstyle,
    ]{definition, notation, algorithm}
\fi
\declaretheorem[
    numbered=no,
    style=definitionstyle,
]{Definition, Notation, Algorithm}

% STYLE FOR EXAMPLES
\declaretheoremstyle[
    headfont=\bfseries\itshape,
    bodyfont=\normalfont,
    notefont=\bfseries\itshape,
    headpunct={.\hfill\break},
]{examplestyle}
\ifbook
    \declaretheorem[
        numbered=yes,
        style=examplestyle,
        within=chapter,
    ]{exercise, example, ex, counterexample}
\else
    \declaretheorem[
        numbered=yes,
        style=examplestyle,
    ]{exercise, example, ex, counterexample}
\fi
\declaretheorem[
    numbered=no,
    style=examplestyle,
]{Exercise, Example, Ex, Counterexample}

% STYLE FOR REMARKS
\declaretheoremstyle[
    headfont=\bfseries\itshape,
    bodyfont=\normalfont,
    notefont=\bfseries\itshape,
    % headformat={\llap{\NAME} \NUMBER \NOTE},
]{notestyle}
\ifbook
    \declaretheorem[
        numbered=yes,
        style=notestyle,
        within=chapter,
    ]{note, remark, idea, intuition}
\else
    \declaretheorem[
        numbered=yes,
        style=notestyle,
    ]{note, remark, idea, intuition}
\fi
\declaretheorem[
    numbered=no,
    style=notestyle,
]{Note, Remark, Idea, Intuition}

\newenvironment{case}[1][Case]
    {\par\textit{#1:}\hfill\break}
    {}

\newenvironment{code}[1][Code]
    {\textbf{#1:}\quote}
    {\endquote}

\ifalgorithms
    \newcounter{nalg}[chapter]
    \renewcommand{\thenalg}{\thechapter.\arabic{nalg}}
    \DeclareCaptionLabelFormat{algocaption}{\it Algorithm \thenalg}

    \lstnewenvironment{algorithm}[1][]
    {
        \refstepcounter{nalg}
        \captionsetup{labelformat=algocaption,labelsep=colon}
        \lstset{
            mathescape=true,
            frame=tB,
            numbers=left,
            numberstyle=\tiny,
            basicstyle=\scriptsize,
            keywordstyle=\color{black}\bfseries\em,
            keywords={,input, output, return, datatype, function, in, if, else, elif, for, foreach, while, not, begin, end, true, false, null, break, continue, let, and, or, }
            numbers=left,
            xleftmargin=.04\textwidth,
            #1
        }
    }
    {}
\else
\fi
