\iftufte  % Tufte settings
    \ifbook
        \documentclass[nobib,justified,openany]{tufte-book}
    \else
        \documentclass[nobib,justified]{tufte-handout}
        \ifsyllabus
            \usepackage{pgf-pie}
        \fi
    \fi

    % part format
    \titleformat{\part}[display]
        {\itshape\Huge}  % format applied to label+text
        {\thepart}  % label
        {0ex}  % horizontal separation between label and title body
        {\itshape\Huge}  % before the title body
        [\vspace{1ex}]  % after the title body
    % section format
    \titleformat{\section}%
        {\itshape\Large}% format applied to label+text
        {\llap{\Large\thesection\phantom{m}}}% label
        {0em}% horizontal separation between label and title body
        {}% before the title body
        []% after the title body
    % subsection format
    \titleformat{\subsection}%
        {\itshape\large}% format applied to label+text
        {\llap{\thesubsection\phantom{m}}}% label
        {0em}% horizontal separation between label and title body
        {}% before the title body
        []% after the title body

    \makeatletter
    % Paragraph indentation and separation for normal text
    \renewcommand{\@tufte@reset@par}{%
        \setlength{\RaggedRightParindent}{0.0pc}%
        \setlength{\JustifyingParindent}{0.0pc}%
        \setlength{\parindent}{0.0pc}%
        \setlength{\parskip}{0.25\baselineskip}%
    }
    \@tufte@reset@par

    % Paragraph indentation and separation for marginal text
    \renewcommand{\@tufte@margin@par}{%
        \setlength{\RaggedRightParindent}{0.0pc}%
        \setlength{\JustifyingParindent}{0.0pc}%
        \setlength{\parindent}{0.0pc}%
        \setlength{\parskip}{0.25\baselineskip}%
    }
    \makeatother

    % TODO: this for margintable & put in preamble
    % Remove \FloatBarrier from marginfigure definition to prevent unwanted spaces.
    \makeatletter
    \renewenvironment{@tufte@margin@float}[2][-1.2ex]{%
        %\FloatBarrier% removed because it adds unwanted white space
        \begin{lrbox}{\@tufte@margin@floatbox}%
            \begin{minipage}{\marginparwidth}%
                \@tufte@caption@font
                \def\@captype{#2}%
                \hbox{}\vspace*{#1}%
                \@tufte@caption@justification
                \@tufte@margin@par
                \noindent
            }{%
            \end{minipage}%
        \end{lrbox}%
        \marginpar{\usebox{\@tufte@margin@floatbox}}%
    }
    \makeatother

    \usepackage{microtype}
    \usepackage{arydshln, chngcntr}
    \setcounter{tocdepth}{1}
    \setcounter{secnumdepth}{2}

    \usepackage{tipa}
    \usepackage{fancyvrb}
    \renewcommand{\figurename}{Fig.}
    \renewcommand{\tablename}{Tab.}
\else  % regular LaTeX settings
    \ifbook
        \documentclass[letterpaper]{book}
    \else
        \documentclass[letterpaper]{article}
    \fi
    \usepackage[left=2cm,top=2cm,right=2cm,bottom=2cm,bindingoffset=0cm]{geometry}
    \usepackage[math]{cellspace}
    \cellspacetoplimit 1pt
    \cellspacebottomlimit 1pt
    \usepackage{hyperref, caption, subcaption}
\fi

\AtBeginDocument{%
    % reset AMS mathbb symbols
    \DeclareSymbolFont{AMSb}{U}{msb}{m}{n}
    \DeclareSymbolFontAlphabet{\mathbb}{AMSb}

    % reset default math fonts
    \DeclareSymbolFont{operators}   {OT1}{cmr} {m}{n}
    \DeclareSymbolFont{letters}     {OML}{cmm} {m}{it}
    \DeclareSymbolFont{symbols}     {OMS}{cmsy}{m}{n}
    \DeclareSymbolFont{largesymbols}{OMX}{cmex}{m}{n}
    \SetSymbolFont{operators}{bold}{OT1}{cmr} {bx}{n}
    \SetSymbolFont{letters}  {bold}{OML}{cmm} {b}{it}
    \SetSymbolFont{symbols}  {bold}{OMS}{cmsy}{b}{n}
    \DeclareSymbolFontAlphabet{\mathrm}    {operators}
    \DeclareSymbolFontAlphabet{\mathnormal}{letters}
    \DeclareSymbolFontAlphabet{\mathcal}   {symbols}
    \DeclareMathAlphabet      {\mathbf}{OT1}{cmr}{bx}{n}
    \DeclareMathAlphabet      {\mathsf}{OT1}{cmss}{m}{n}
    \DeclareMathAlphabet      {\mathit}{OT1}{cmr}{m}{it}
    \DeclareMathAlphabet      {\mathtt}{OT1}{cmtt}{m}{n}
}

% math
\usepackage{amsmath, amsfonts, amssymb, amstext, amscd, amsthm, bm, faktor, mathrsfs, mathtools, mdframed, thmtools, xfrac}
% fonts
\usepackage{bbm, CJKutf8, dsfont, marvosym, stmaryrd}
% tables
\usepackage{booktabs, colortbl, makecell}
% colors
\usepackage{color, soul, xcolor}
% figures
\usepackage{graphicx, float, tikz, pgfplots}
% headers and footers
\usepackage{fancyhdr, lastpage}
% miscellaneous
\usepackage{enumerate, ifthen, lipsum, listings, imakeidx, parskip, ulem, verbatim, xargs}
% references
\usepackage{url}
\usepackage[nodayofweek]{datetime}

\usepackage[group-separator={,},group-minimum-digits={3}]{siunitx}
\usepackage[shortlabels]{enumitem}
\setlist[enumerate]{topsep=0ex,
                    itemsep=0ex,
                    partopsep=1ex,
                    parsep=1ex}
\setlist[itemize]{topsep=0ex,
                  itemsep=0ex,
                  partopsep=1ex,
                  parsep=1ex,
                  label=$\cdot$}

\hypersetup{
    colorlinks=true,
    linkcolor=blue,
    citecolor=blue,
    urlcolor=blue
}
% \hypersetup{
    % colorlinks=true,
    % linkcolor=orange,
    % citecolor=red,
    % urlcolor=blue
% }

\allowdisplaybreaks
\newdateformat{verbosedate}{\ordinal{DAY} of \monthname[\THEMONTH], \THEYEAR}
\verbosedate

\pagestyle{fancy}
% \fancyfoot[C]{--~\thepage~--}
\fancyfoot[C]{\tiny \thepage\ / \pageref*{LastPage}}
\ifbook
    \fancypagestyle{plain}{%
        \fancyhead[L]{}
        \ifdate
            \fancyhead[R]{\today}
        \else
            \fancyhead[R]{}
        \fi
        \renewcommand{\headrulewidth}{0pt}
    }

    \let\cleardoublepage=\clearpage
\else
    \ifsyllabus
        \fancypagestyle{plain}{%
            \fancyhead[L]{}
            \ifdate
                \fancyhead[R]{\today}
            \else
                \fancyhead[R]{}
            \fi
            \renewcommand{\headrulewidth}{0pt}
        }
    \else
        \fancypagestyle{plain}{}
        \renewcommand{\headrulewidth}{0pt}
    \fi
\fi

\ifdaggerfootnotes
    \renewcommand{\thefootnote}{\fnsymbol{footnote}}
\else
\fi

\delimitershortfall=-1pt
\normalem

\usepackage{epigraph}
\usepackage{etoolbox}
\makeatletter
% \newlength\epitextskip
\pretocmd{\@epitext}{\em}{}{}
\apptocmd{\@epitext}{\em}{}{}
\patchcmd{\epigraph}{\@epitext{#1}\\}{\vspace{-10ex}\@epitext{#1}\\}{}{}
\makeatother

\setlength\epigraphrule{0pt}
% \setlength\epitextskip{0ex}
\setlength\epigraphwidth{.75\textwidth}
