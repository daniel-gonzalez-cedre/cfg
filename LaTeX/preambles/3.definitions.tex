% TODO: work on this later
\AtBeginDocument{
  \let\oldnameref\nameref
  \let\nameref\relax
  \newcommand{\nameref}[1]{\emph{\oldnameref*{#1}}}
}

% to complement \textquotesingle
\newcommand{\textquoteback}{\`{}}

% left margin notes
\long\def\gutternote#1{%
  % \vspace{-0.65\baselineskip}\vadjust{\llap{\smash{\vtop{%
  \vspace{-2ex}\vadjust{\llap{\smash{\vtop{%
    \parindent=0pt
    \hsize=0.6in
    \parfillskip=0pt
    \leftskip=0pt
    \baselineskip=0pt\flushleft\footnotesize\itshape\vglue-\ht\strutbox #1}}\kern1em}}%
% \vspace{0.65\baselineskip}}
\vspace{2ex}}

% for horizontally shifting a symbol
\makeatletter
\newcommand*{\shifttext}[2]{%
  \settowidth{\@tempdima}{#2}%
  \makebox[\@tempdima]{\hspace*{#1}#2}%
}
\makeatother

\newcommand{\blankpage}{\newpage\hbox{}\thispagestyle{empty}\newpage}
\newcommand*{\xline}[1][3em]{\rule[0.5ex]{#1}{0.55pt}}

\newcommand{\bydefn}{\DOTSB\;\mathrel{\vcentcolon\Leftrightarrow}\;}
\newcommand{\iffbydefn}{\DOTSB\;\mathrel{\vcentcolon\Leftrightarrow}\;}
\newcommand{\defiff}{\DOTSB\;\mathrel{\vcentcolon\Leftrightarrow}\;}

\newcommand{\proves}{\DOTSB\:\vdash\:}
\newcommand{\satisfies}{\DOTSB\:\models\:}
% \renewcommand{\implies}{\DOTSB\,\Rightarrow\,}
\renewcommand{\implies}{\DOTSB\:\mathrel{\Rightarrow}\:}
% \renewcommand{\impliedby}{\DOTSB\,\Leftarrow\,}
\renewcommand{\impliedby}{\DOTSB\:\mathrel{\Leftarrow}\:}
% \renewcommand{\iff}{\DOTSB\,\Leftrightarrow\,}
\renewcommand{\iff}{\mathrel{\DOTSB\:\Leftrightarrow}\:}
\newcommand{\lequiv}{\DOTSB\:\equiv\:}
% \newcommand{\isomorphic}{\DOTSB\,\cong\,}
\newcommand{\isomorphic}{\mathrel{\cong}}

\newcommand{\notproves}{\DOTSB\,\nvdash\,}
\newcommand{\notsatisfies}{\DOTSB\,\not\models\,}
\newcommand{\notlequiv}{\DOTSB\,\not\equiv\,}
\newcommand{\notimplies}{\DOTSB\,\not\Rightarrow\,}
\newcommand{\notimpliedby}{\DOTSB\,\not\Leftarrow\,}
\newcommand{\notiff}{\mathrel{{\ooalign{\hidewidth$\not\phantom{"}$\hidewidth\cr$\iff$}}}}
\newcommand{\notisomorphic}{\DOTSB\,\not\cong\,}

\renewcommand{\qedsymbol}{\scshape q.e.d.}

\renewcommand{\leq}{\leqslant}
\renewcommand{\geq}{\geqslant}

\let\oldtop\top
\let\oldbot\bot
\renewcommand{\top}{\vcenter{\hbox{$\oldtop$}}}
% \renewcommand{\bot}{\vcenter{\hbox{$\oldbot$}}}

% NOR operation
\makeatletter
  \providecommand*{\barvee}{\mathbin{\mathpalette\@barvee{}}}
  \newcommand*{\@barvee}[2]{%
    \sbox0{$#1\veebar\m@th$}%
    \sbox2{\hbox to \wd0 {\hss\resizebox{1.05\wd0}{\height}{$#1-\m@th$}\hss}}
    \sbox4{\resizebox{\wd0}{.7\ht0}{$#1\vee\m@th$}}%
    \sbox6{$#1\vcenter{}$}
    \ht2=\ht6
    \vbox to \ht0{\copy2\vss\copy4}
  }
\makeatother

% string concatenation operation
\makeatletter
\newcommand{\concat}{\mathbin{\mathpalette\conc@t\relax}}
\newcommand{\conc@t}[2]{%
  \vcenter{\hbox{%
    \sbox\z@{$\m@th#1-$}%
    \setlength{\unitlength}{\wd\z@}%
    \begin{picture}(1,1)
    \roundcap
    \put(0.1,0.5){\line(1,0){0.8}}
    \put(0.35,0.1){\line(0,1){0.8}}
    \put(0.65,0.1){\line(0,1){0.8}}
    \end{picture}%
  }}%
}
\makeatother

\newcommand{\conditional}{\rightarrow}
\newcommand{\biconditional}{\leftrightarrow}

\renewcommand{\land}{\mathrel{\wedge}}
\renewcommand{\lor}{\mathrel{\vee}}
\newcommand{\lif}{\mathrel{\rightarrow}}
\newcommand{\lifthen}{\mathrel{\rightarrow}}
\newcommand{\liff}{\mathrel{\leftrightarrow}}

\newcommand{\Land}{\DOTSB\,\wedge\,}
\newcommand{\Lor}{\DOTSB\,\vee\,}
\newcommand{\Lif}{\DOTSB\,\rightarrow\,}
\newcommand{\Liff}{\DOTSB\,\leftrightarrow\,}

\newcommand{\bigand}{\bigwedge}
\newcommand{\bigor}{\bigwedge}

\newcommand{\meet}{\wedge}
\newcommand{\join}{\vee}
\newcommand{\bigmeet}{\bigwedge}
\newcommand{\bigjoin}{\bigvee}

\newcommand{\conjunct}{\wedge}
\newcommand{\disjunct}{\vee}
\newcommand{\bigconjunct}{\bigwedge}
\newcommand{\bigdisjunct}{\bigvee}

\newcommand{\defeq}{\mathrel{\coloneqq}}
\newcommand{\xor}{\mathrel{\veebar}}
\newcommand{\nand}{\mathrel{\barwedge}}
\newcommand{\nor}{\mathrel{\barvee}}
\newcommand{\Plus}{\text{\begin{tikzpicture}[baseline=-0.56ex, x=1ex, y=1ex]
    \node at (-26/40, 0) {};
    \node at (26/40, 0) {};
    \draw [line width=1.25, line cap=round] (0, -2/3) -- (0, 2/3);
    \draw [line width=1.25, line cap=round] (-2/3, 0) -- (2/3, 0);
\end{tikzpicture}}}
\newcommand{\tetrate}{\mathrel{\uparrow\uparrow}}
\newcommand{\tetration}{\mathrel{\uparrow\uparrow}}


\newcommand{\comp}{\circ}
\newcommand{\compose}{\circ}
\renewcommand{\div}{\mid}
\newcommand{\ndiv}{\nmid}
\newcommand{\divides}{\mathrel{\mid}}
\newcommand{\notdivides}{\nmid}
\newcommand{\given}{\;\vert\;}
\newcommand{\suchthat}{\;\vert\;}
% \newcommand{\contradiction}{\smash{\textnormal{\raisebox{-0.25ex}{\Large \Lightning}}} }
\newcommand{\contradiction}{\smash{\textnormal{\faBolt}}\, }

\newcommand{\mean}[1]{\overline{#1}}
\newcommand{\conjugate}[1]{\overline{#1}}
% \newcommand{\complement}[1]{\overline{#1}}  % already defined lol?

\newcommand*\diff{\mathop{}\!\mathrm{d}}
\newcommand{\integral}[1]{\smashoperator{\int_{#1}}}
\newcommand{\der}[2]{\frac{\diff{#1}}{\diff{#2}}}
\newcommand{\dern}[3]{\frac{\diff^{#3}{#1}}{\diff{#2}^{#3}}}
\newcommand{\derm}[3]{\frac{\diff^{#3}{#1}}{\diff{#2}}}
\newcommand{\prt}[2]{\frac{\partial{#1}}{\partial{#2}}}
\newcommand{\prtn}[3]{\frac{\partial^{#3}{#1}}{\partial{#2}^{#3}}}
\newcommand{\prtm}[3]{\frac{\partial^{#3}{#1}}{\partial{#2}}}

\DeclarePairedDelimiterXPP{\E}[1]{\mathbb{E}}{[}{]}{}{#1}
\DeclarePairedDelimiterXPP{\Esub}[2]{\mathbb{E}}{[}{]}{#1}{#2}
% \newcommand{\E}[1]{\mathbb{E}\sq*{#1}}
% \newcommand{\Esub}[2]{\mathbb{E}_{#1}\sq*{#2}}
\DeclarePairedDelimiterXPP{\var}[1]{\mathrm{Var}}{(}{)}{}{#1}
\DeclarePairedDelimiterXPP{\cov}[2]{\mathrm{Cov}}{(}{)}{}{#1, #2}
% \newcommand{\var}[1]{\mathrm{Var}\term*{#1}}
% \newcommand{\cov}[2]{\mathrm{Cov}\term*{#1, #2}}

% better modular congruence
\makeatletter
  \DeclarePairedDelimiterX{\pmodx}[1]{(}{)}{{\operator@font mod}\mkern6mu#1}
  \renewcommand{\pmod}{%
    \allowbreak
    \if@display\mkern18mu\else\mkern8mu\fi
    \pmodx
  }
  \DeclarePairedDelimiterXPP{\modulo}[3]{%
    {#1}\equiv{#2}
    \allowbreak
    \if@display\mkern18mu\else\mkern8mu\fi
  }{(}{)}{}{{\operator@font mod}\mkern6mu#3}
  \DeclarePairedDelimiterXPP{\congruent}[3]{%
    {#1}\equiv{#2}
    \allowbreak
    \if@display\mkern18mu\else\mkern8mu\fi
  }{(}{)}{}{{\operator@font mod}\mkern6mu#3}
  \DeclarePairedDelimiterXPP{\notcongruent}[3]{%
    {#1}\not\equiv{#2}
    \allowbreak
    \if@display\mkern18mu\else\mkern8mu\fi
  }{(}{)}{}{{\operator@font mod}\mkern6mu#3}
\makeatother

\newcommand{\quotient}[2]{\ensuremath{{#1}/{#2}}}
\newcommand{\zmod}[1]{\ensuremath{\mathbb{Z}/{#1}\mathbb{Z}}}
\newcommand{\Z}[1]{\ensuremath{\sfrac{\mathbb{Z}}{{#1}\mathbb{Z}}}}
\newcommand{\z}[1]{\ensuremath{\sfrac{\mathbb{Z}}{{#1}\mathbb{Z}}}}

\newcommand{\chr}[1]{\texttt{#1}}
\newcommand{\str}[1]{\texttt{"#1"}}
\newcommand{\substr}[2]{\text{\normalshape \tt [}\ifblank{#1}{\!}{#1}:\ifblank{#2}{\!}{#2}\text{\normalshape \tt ]}}
\newcommand{\slice}[2]{\text{\normalshape \tt [}\ifblank{#1}{\!}{#1}:\ifblank{#2}{\!}{#2}\text{\normalshape \tt ]}}

% \DeclarePairedDelimiterXPP{\slice}[2]{}{\texttt{[}}{\texttt{]}}{}{\ifblank{#1}{\!}{#1}\!:\!\ifblank{#2}{\!}{#2}}

\renewcommand{\restriction}[1]{\downharpoonright_{#1}}
\newcommand{\restricted}[2]{#1\downharpoonright_{#2}}
\newcommand*\id{\text{\normalshape \tt id}}

\let\oldker\ker
\let\ker\relax
\DeclarePairedDelimiterXPP{\ker}[1]{\text{\normalshape \tt ker}}{(}{)}{}{#1}
\DeclarePairedDelimiterXPP{\ran}[1]{\text{\normalshape \tt ran}}{(}{)}{}{#1}

\newcommand{\inj}{\hookrightarrow}
\newcommand{\inject}{\hookrightarrow}
\newcommand{\injection}{\hookrightarrow}

\newcommand{\surj}{\twoheadrightarrow}
\newcommand{\surject}{\twoheadrightarrow}
\newcommand{\surjection}{\twoheadrightarrow}

\newcommand{\bij}{\lhook\joinrel\twoheadrightarrow}
\newcommand{\biject}{\lhook\joinrel\twoheadrightarrow}
\newcommand{\bijection}{\lhook\joinrel\twoheadrightarrow}

\newcommand{\monic}{\hookrightarrow}
\newcommand{\monomorphism}{\hookrightarrow}

\newcommand{\epic}{\twoheadrightarrow}
\newcommand{\epimorphism}{\twoheadrightarrow}

\newcommand{\iso}{\lhook\joinrel\twoheadrightarrow}
\newcommand{\isomorphism}{\lhook\joinrel\twoheadrightarrow}
\newcommand{\immersion}{\looprightarrow}

\let\olds\s
\let\s\relax
\DeclarePairedDelimiterXPP{\s}[1]{\mathfrak{s}}{(}{)}{}{#1}
\let\oldS\S
\let\S\relax
\DeclarePairedDelimiterXPP{\S}[1]{\mathcal{S}}{(}{)}{}{#1}
\let\oldO\O
\let\O\relax
\DeclarePairedDelimiterXPP{\O}[1]{\mathcal{O}}{(}{)}{}{#1}
\let\oldP\P
\let\P\relax
\DeclarePairedDelimiterXPP{\P}[1]{\mathcal{P}}{(}{)}{}{#1}
\let\oldP\P
\let\P\relax
\DeclarePairedDelimiterXPP{\F}[1]{\mathcal{F}}{(}{)}{}{#1}
\DeclarePairedDelimiterXPP{\fib}[1]{\mathcal{F}}{(}{)}{}{#1}
\DeclarePairedDelimiterXPP{\fibonacci}[1]{\mathcal{F}}{(}{)}{}{#1}
\DeclarePairedDelimiterXPP{\successor}[1]{\textsc{suc}}{(}{)}{}{#1}
\DeclarePairedDelimiterXPP{\power}[1]{\mathbb{P}}{(}{)}{}{#1}

\newcommand{\graph}[1]{#1}
\newcommand{\vertices}[1]{\textnormal{\texttt{V}}_{#1}}
\newcommand{\nodes}[1]{\textnormal{\texttt{V}}_{#1}}
\newcommand{\edges}[1]{\textnormal{\texttt{E}}_{#1}}
\DeclarePairedDelimiterXPP{\neighborhood}[2]{\textnormal{\texttt{N}}_{#1}}{(}{)}{}{#2}
\DeclarePairedDelimiterXPP{\incidence}[2]{\textnormal{\texttt{I}}_{#1}}{(}{)}{}{#2}

\let\olddeg\deg
\let\deg\relax
\DeclarePairedDelimiterXPP{\deg}[1]{\texttt{deg}}{(}{#1}{)}{}
\DeclarePairedDelimiterXPP{\degree}[2]{\texttt{deg}_{#1}}{(}{#2}{)}{}

\newcommand{\naturals}{\mathbb{N}}
\newcommand{\integers}{\mathbb{Z}}
\newcommand{\rationals}{\mathbb{Q}}
\newcommand{\reals}{\mathbb{R}}
\newcommand{\complex}{\mathbb{C}}

% these don't need {} after them since they should be followed by text
\newcommand{\cf}{\textit{cf.},\ }
\newcommand{\eg}{\textit{e.g.},\ }
\newcommand{\ie}{\textit{i.e.},\ }
% \newcommand{\etc}{\textit{etc.}\ }  % TODO: already defined? check this
\newcommand{\aka}{\textit{a.k.a.}\ }
\newcommand{\sic}{\textit{sic.}\ }
\newcommand{\viz}{\textit{viz.}\ }
\newcommand{\vide}{\textit{v.}\ }
\newcommand{\qv}{\textit{q.v.}\ }
\newcommand{\vv}{\textit{vice versa}}
\newcommand{\ca}{ca.\ }
\newcommand{\circa}{ca.\ }
\newcommand{\ifandonlyif}{\textit{iff}\ }
\newcommand{\textiff}{\textit{iff}\ }

% these need {} after them
\newcommand*{\etal}{\textit{et al.}}
\newcommand*{\wff}{\textit{wff}}

\DeclareMathOperator{\lcm}{lcm}
\DeclareMathOperator*{\argmin}{arg\!\min}
\DeclareMathOperator*{\argmax}{arg\!\max}

\let\originalleft\left
\let\originalright\right
\renewcommand{\left}{\mathopen{}\mathclose\bgroup\originalleft}
\renewcommand{\right}{\aftergroup\egroup\originalright}

% chinese and japanese fonts
\newcommand{\zh}[1]{\begin{CJK}{UTF8}{min}#1\end{CJK}}
\newcommand{\hanzi}[1]{\begin{CJK}{UTF8}{gkai}#1\end{CJK}}
\newcommand{\kanji}[1]{\begin{CJK}{UTF8}{bkai}#1\end{CJK}}
\newcommand{\kana}[1]{\begin{CJK}{UTF8}{min}#1\end{CJK}}

% parentheses and delimiters
\DeclarePairedDelimiterX \abs[1]{\lvert}{\rvert}{\ifblank{#1}{\:\cdot\:}{#1}}
\DeclarePairedDelimiterX \norm[1]{\lVert}{\rVert}{\ifblank{#1}{\:\cdot\:}{#1}}
\DeclarePairedDelimiterX \inner[2]{\langle}{\rangle}{\ifblank{#1}{\:\cdot\:}{#1},\ifblank{#2}{\:\cdot\:}{#2}}
\DeclarePairedDelimiterX \bracket[1]{\langle}{\rangle}{\ifblank{#1}{\:\cdot\:}{#1}}
\DeclarePairedDelimiterX \bra[1]{\langle}{\rvert}{\ifblank{#1}{\:\cdot\:}{#1}}
\DeclarePairedDelimiterX \ket[1]{\lvert}{\rangle}{\ifblank{#1}{\:\cdot\:}{#1}}
\DeclarePairedDelimiterX \floor[1]{\lfloor}{\rfloor}{\ifblank{#1}{\:\cdot\:}{#1}}
\DeclarePairedDelimiterX \ceil[1]{\lceil}{\rceil}{\ifblank{#1}{\:\cdot\:}{#1}}
\DeclarePairedDelimiterX \ceiling[1]{\lceil}{\rceil}{\ifblank{#1}{\:\cdot\:}{#1}}

\newcommand{\falling}[1]{^{\underline{\ifblank{#1}{\:\cdot\!\;}{#1}}}}
\newcommand{\rising}[1]{^{\overline{\ifblank{#1}{\:\cdot\!\;}{#1}}}}

\DeclarePairedDelimiter \card{\lvert}{\rvert}
\DeclarePairedDelimiter \cardinality{\lvert}{\rvert}
\DeclarePairedDelimiter \order{\lvert}{\rvert}
\DeclarePairedDelimiter \size{\lVert}{\rVert}

\DeclarePairedDelimiterX \set[1]{\lbrace}{\rbrace}{%
  \renewcommand{\suchthat}{\;\delimsize\vert\nonscript\;\mathopen{}}%
  \renewcommand{\given}{\;\delimsize\vert\nonscript\;\mathopen{}}#1%
}
\DeclarePairedDelimiterX \term[1]{(}{)}{%
  \renewcommand{\suchthat}{\;\delimsize\vert\nonscript\;\mathopen{}}%
  \renewcommand{\given}{\;\delimsize\vert\nonscript\;\mathopen{}}#1%
}

\DeclarePairedDelimiter \pn{(}{)}
\DeclarePairedDelimiter \parens{(}{)}

\DeclarePairedDelimiter \braces{\{}{\}}
\DeclarePairedDelimiter \curly{\{}{\}}

\DeclarePairedDelimiter \sq{\lbrack}{\rbrack}
\DeclarePairedDelimiter \brackets{\lbrack}{\rbrack}
\DeclarePairedDelimiter \range{\lbrack}{\rbrack}
\DeclarePairedDelimiter \class{\lbrack}{\rbrack}
\DeclarePairedDelimiterXPP{\eqclass}[2]{}{\lbrack}{\rbrack}{_{#2}}{#1}
\DeclarePairedDelimiter \nat{\llbracket}{\rrbracket}  % for natural numbers

\DeclarePairedDelimiter \seq{\langle}{\rangle}
\DeclarePairedDelimiter \sequence{\langle}{\rangle}

\let\oldemptyset\emptyset
\let\emptyset\varnothing
\let\union\cup
\let\Union\bigcup
\let\intersection\cap
\let\intersect\cap
\let\Intersection\bigcap
\let\Intersect\bigcap

% TODO: maybe implement this
% \newcommand{\smashsum}[2][]{\smashoperator{\sum_{\ifblank{#1}{#1}{#1}}^{\ifblank{#2}}}}

% an asterisk for hanging above punctuation
\newcommand{\hangasterisk}{\makebox[0pt][l]{*}}  % before punctuation
% \newcommand{\hangasterisk}{\makebox[1pt][r]{*}}  % after punctuation with {} around

% stacked asterisk symbol \stars{n}
\makeatletter
  \newcommand{\stars}{}% just for safety
  \DeclareRobustCommand{\stars}[1]{\stars@{#1}}
  \newcommand{\stars@}[1]{%
    \ifcase#1\relax\or\stars@one\or\stars@two\or\stars@three\or\stars@four\or\stars@five\or\stars@six
    \else ??\fi
  }
  \newcommand{\stars@char}{\text{*}}
  \newcommand{\stars@base}[1]{%
    $\m@th\vcenter{\offinterlineskip\lineskip=-0.65ex\ialign{\hfil##\hfil\cr#1\crcr}}$%
  }
  \newcommand{\stars@one}{%
    \stars@base{\stars@char}%
  }
  \newcommand{\stars@two}{%
    \stars@base{\stars@char\cr\stars@char}%
  }
  \newcommand{\stars@three}{%
    \stars@base{\stars@char\cr\stars@char\stars@char}%
  }
  \newcommand{\stars@four}{%
    \stars@base{\stars@char\stars@char\cr\stars@char\stars@char}%
  }
  \newcommand{\stars@five}{%
    \stars@base{\stars@char\stars@char\cr\stars@char\stars@char\stars@char}%
  }
  \newcommand{\stars@six}{%
    \stars@base{\stars@char\cr\stars@char\stars@char\cr\stars@char\stars@char\stars@char}%
  }
\makeatother
