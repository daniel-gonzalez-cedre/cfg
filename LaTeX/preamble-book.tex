\documentclass[]{book}

\usepackage{amsmath, amsfonts, amssymb, amstext, amscd, amsthm, arydshln, bbm, booktabs, CJKutf8, caption, chngcntr, color, colortbl, dsfont, enumerate, float, graphicx, hyperref, lipsum, listings, makecell, makeidx, mathrsfs, mathtools, marvosym, parskip, soul, stmaryrd, tikz, url, verbatim, xcolor, xfrac}
\usepackage[left=2cm,top=2cm,right=2cm,bottom=2cm,bindingoffset=0cm]{geometry}
\usepackage[group-separator={,},group-minimum-digits={3}]{siunitx}
\usepackage[math]{cellspace}
\cellspacetoplimit 6pt
\cellspacebottomlimit 6pt
\hypersetup{
    citecolor=black!50!red,
    colorlinks=true,
    linkcolor=black!50!red,
    urlcolor=black!25!blue
}

% define new conditionals: \if<conditional>
\newif\ifindentproofs
\newif\ifindenttheorems

% set the values of those conditionals: \<conditional>true / \<conditional>false
\indentproofstrue
\indenttheoremstrue

\allowdisplaybreaks

\newtheoremstyle{thm}
  {}%           Space above, empty = `usual value'
  {}%           Space below
  {\itshape}%   Body font
  {}%           Indent amount (empty = no indent, \parindent = para indent)
  {\bfseries}%  Thm head font
  {}%           Punctuation after thm head
  {\newline}%   Space after thm head: \newline = linebreak
  {}%           Thm head spec
\newtheoremstyle{dfn}
  {}%           Space above, empty = `usual value'
  {}%           Space below
  {}%           Body font
  {}%           Indent amount (empty = no indent, \parindent = para indent)
  {\bfseries}%  Thm head font
  {}%           Punctuation after thm head
  {\newline}%   Space after thm head: \newline = linebreak
  {}%           Thm head spec
\newtheoremstyle{rmk}
  {}%           Space above, empty = `usual value'
  {}%           Space below
  {}%           Body font
  {}%           Indent amount (empty = no indent, \parindent = para indent)
  {\bfseries}%  Thm head font
  {}%           Punctuation after thm head
  {\newline}%   Space after thm head: \newline = linebreak
  {}%           Thm head spec

\ifindentproofs
    % begin new proof environment
    \expandafter\let\expandafter\oldproof\csname\string\proof\endcsname
    \let\oldendproof\endproof

    \renewenvironment{proof}[1][\proofname]{%
        \oldproof[#1]\quote~

    }{%
        \endquote\oldendproof
    }
    % end new proof environment
\else
\fi

\ifindenttheorems
    \theoremstyle{thm}% style for theorems
    \newtheorem{pre-theorem}{Theorem}
    \newtheorem{pre-lemma}{Lemma}
    \newtheorem{pre-proposition}{Proposition}
    \newtheorem{pre-corollary}{Corollary}
    \newtheorem{pre-claim}{Claim}
    \newtheorem{pre-conjecture}{Conjecture}
    \newtheorem{pre-justification}{Justification}

    \theoremstyle{dfn}% style for definitions
    \newtheorem{pre-axiom}{Axiom}
    \newtheorem{pre-definition}{Definition}
    \newtheorem{pre-exercise}{Exercise}
    \newtheorem{pre-example}{Example}
    \newtheorem{pre-counterexample}{Counterexample}

    \theoremstyle{rmk}% style for remarks
    \newtheorem{pre-idea}{Idea}
    \newtheorem*{pre-remark}{Remark}
    \newtheorem*{pre-note}{Note}

    % theorem
    \newenvironment{theorem}[1][]
        {\begin{pre-theorem}[#1]\quote~

            }
        {\endquote\end{pre-theorem}}
    % lemma
    \newenvironment{lemma}[1][]
        {\begin{pre-lemma}[#1]\quote~

            }
        {\endquote\end{pre-lemma}}
    % proposition
    \newenvironment{proposition}[1][]
        {\begin{pre-proposition}[#1]\quote~

            }
        {\endquote\end{pre-proposition}}
    % corollary
    \newenvironment{corollary}[1][]
        {\begin{pre-corollary}[#1]\quote~

            }
        {\endquote\end{pre-corollary}}
    % claim
    \newenvironment{claim}[1][]
        {\begin{pre-claim}[#1]\quote~

            }
        {\endquote\end{pre-claim}}
    % conjecture
    \newenvironment{conjecture}[1][]
        {\begin{pre-conjecture}[#1]\quote~

            }
        {\endquote\end{pre-conjecture}}
    % justification
    \newenvironment{justification}[1][]
        {\begin{pre-justification}[#1]\quote~

            }
        {\endquote\end{pre-justification}}

    % axiom
    \newenvironment{axiom}[1][]
        {\begin{pre-axiom}[#1]\quote~

            }
        {\endquote\end{pre-axiom}}
    % definition
    \newenvironment{definition}[1][]
        {\begin{pre-definition}[#1]\quote~

            }
        {\endquote\end{pre-definition}}
    % exercise
    \newenvironment{exercise}[1][]
        {\begin{pre-exercise}[#1]\quote~

            }
        {\endquote\end{pre-exercise}}
    % example
    \newenvironment{example}[1][]
        {\begin{pre-example}[#1]\quote~

            }
        {\endquote\end{pre-example}}
    % counterexample
    \newenvironment{counterexample}[1][]
        {\begin{pre-counterexample}[#1]\quote~

            }
        {\endquote\end{pre-counterexample}}

    % idea
    \newenvironment{idea}[1][]
        {\begin{pre-idea}[#1]\quote~

            }
        {\endquote\end{pre-idea}}
    % remark
    \newenvironment{remark}[1][]
        {\begin{pre-remark}[#1]\quote~

            }
        {\endquote\end{pre-remark}}
    % note
    \newenvironment{note}[1][]
        {\begin{pre-note}[#1]\quote~

            }
        {\endquote\end{pre-note}}
\else
    \theoremstyle{thm}% style for theorems
    \newtheorem{theorem}{Theorem}
    \newtheorem{lemma}{Lemma}
    \newtheorem{proposition}{Proposition}
    \newtheorem{corollary}{Corollary}
    \newtheorem{claim}{Claim}
    \newtheorem{conjecture}{Conjecture}
    \newtheorem{justification}{Justification}

    \theoremstyle{dfn}% style for definitions
    \newtheorem{axiom}{Axiom}
    \newtheorem{definition}{Definition}
    \newtheorem{exercise}{Exercise}
    \newtheorem{example}{Example}
    \newtheorem{counterexample}{Counterexample}

    \theoremstyle{rmk}% style for remarks
    \newtheorem{idea}{Idea}
    \newtheorem*{remark}{Remark}
    \newtheorem*{note}{Note}
\fi

\def\lstlistingautorefname{Algorithm}
\def\itemautorefname{Section}
\newcommand{\lemmaautorefname}{Lemma}
\newcommand{\propositionautorefname}{Proposition}
\newcommand{\corollaryautorefname}{Corollary}
\newcommand{\claimautorefname}{Claim}
\newcommand{\conjectureautorefname}{Conjecture}
\newcommand{\justificationautorefname}{Justification}
\newcommand{\definitionautorefname}{Definition}
\newcommand{\exampleautorefname}{Example}
\newcommand{\counterexampleautorefname}{Counterexample}
\newcommand{\ideaautorefname}{Idea}

\newcommand*{\xline}[1][3em]{\rule[0.5ex]{#1}{0.55pt}}

\newcommand{\defniff}{\mathrel{\vcentcolon\Leftrightarrow}}
\newcommand{\iffbydefn}{ $ \defniff $ }
\newcommand{\niff}{\mathrel{{\ooalign{\hidewidth$\not\phantom{"}$\hidewidth\cr$\iff$}}}}
\renewcommand{\restriction}[1]{\downharpoonright_{#1}}
\renewcommand{\qedsymbol}{\sc q.e.d.}
\renewcommand{\leq}{\leqslant}
\renewcommand{\geq}{\geqslant}
\newcommand{\conjunct}{\wedge}
\newcommand{\disjunct}{\vee}
\newcommand{\defn}{\coloneqq}
\newcommand{\xor}{~\oplus~}
\newcommand{\nand}{~\uparrow~}
\newcommand{\nor}{~\downarrow~}
\newcommand{\divides}{~|~}
\newcommand{\given}{~\middle|~}
\newcommand{\suchthat}{~\middle|~}
\newcommand{\contradiction}{~\text{\Large \Lightning}}
\newcommand{\conj}[1]{\overline{#1}}
\newcommand{\mean}[1]{\overline{#1}}
\newcommand{\integral}[1]{\smashoperator{\int_{#1}}}
\newcommand*\diff{\mathop{}\!\mathrm{d}}
\newcommand{\E}[1]{\mathbb{E}\crochets*{#1}}
\newcommand{\Esub}[2]{\mathbb{E}_{#1}\crochets*{#2}}
\newcommand{\var}[1]{\mathrm{Var}\parens*{#1}}
\newcommand{\cov}[2]{\mathrm{Cov}\parens*{#1, #2}}
\newcommand{\der}[2]{\frac{\diff{#1}}{\diff{#2}}}
\newcommand{\dern}[3]{\frac{\diff^{#3}{#1}}{\diff{#2}^{#3}}}
\newcommand{\derm}[3]{\frac{\diff^{#3}{#1}}{\diff{#2}}}
\newcommand{\prt}[2]{\frac{\partial{#1}}{\partial{#2}}}
\newcommand{\prtn}[3]{\frac{\partial^{#3}{#1}}{\partial{#2}^{#3}}}
\newcommand{\prtm}[3]{\frac{\partial^{#3}{#1}}{\partial{#2}}}
\newcommand{\Mod}[1]{~\parens{\mathrm{mod}~#1}}

\renewcommand{\O}[1]{\mathcal{O}\parens*{#1}}
\newcommand{\C}{\mathbb{C}}
\newcommand{\N}{\mathbb{N}}
\newcommand{\Q}{\mathbb{Q}}
\newcommand{\R}{\mathbb{R}}
\newcommand{\Z}{\mathbb{Z}}

\newcommand{\cf}{\textit{c.f.}, }
\newcommand{\eg}{\textit{e.g.}, }
\newcommand{\ie}{\textit{i.e.}, }
\newcommand{\viz}{\textit{viz. }}
\newcommand{\vide}{\textit{v. }}
\newcommand{\etal}{\textit{et al. }}

\DeclareMathOperator{\lcm}{lcm}
\DeclareMathOperator*{\argmin}{arg\!\min}
\DeclareMathOperator*{\argmax}{arg\!\max}

\let\originalleft\left
\let\originalright\right
\renewcommand{\left}{\mathopen{}\mathclose\bgroup\originalleft}
\renewcommand{\right}{\aftergroup\egroup\originalright}
\newcommand{\zh}[1]{\begin{CJK}{UTF8}{gbsn}#1\end{CJK}}
\newcommand{\jp}[1]{\begin{CJK}{UTF8}{gbsn}#1\end{CJK}}

\DeclarePairedDelimiterX \inner[2]{\langle}{\rangle}{#1,#2}
\DeclarePairedDelimiter \bra{\langle}{\rvert}
\DeclarePairedDelimiter \ket{\lvert}{\rangle}
\DeclarePairedDelimiter \abs{\lvert}{\rvert}
\DeclarePairedDelimiter \norm{\lVert}{\rVert}
\DeclarePairedDelimiter \set{\lbrace}{\rbrace}
\DeclarePairedDelimiter \seq{\langle}{\rangle}
\DeclarePairedDelimiter \parens{(}{)}
\DeclarePairedDelimiter \crochets{[}{]}
\DeclarePairedDelimiter \brackets{\langle}{\rangle}

\let\oldemptyset\emptyset
\let\emptyset\varnothing
\let\union\cup
\let\intersection\cap
\let\intersect\cap

\renewcommand{\theequation}{\thechapter.\arabic{equation}}
\renewcommand{\thetheorem}{\thechapter.\arabic{theorem}}
\renewcommand{\thelemma}{\thechapter.\arabic{lemma}}
\renewcommand{\theproposition}{\thechapter.\arabic{proposition}}
\renewcommand{\thecorollary}{\thechapter.\arabic{corollary}}
\renewcommand{\theclaim}{\thechapter.\arabic{claim}}
\renewcommand{\theconjecture}{\thechapter.\arabic{conjecture}}
\renewcommand{\thejustification}{\thechapter.\arabic{justification}}
\renewcommand{\thedefinition}{\thechapter.\arabic{definition}}
\renewcommand{\theexample}{\thechapter.\arabic{example}}
\renewcommand{\thecounterexample}{\thechapter.\arabic{counterexample}}

\counterwithin*{equation}{chapter}
\counterwithin*{theorem}{chapter}
\counterwithin*{lemma}{chapter}
\counterwithin*{proposition}{chapter}
\counterwithin*{corollary}{chapter}
\counterwithin*{claim}{chapter}
\counterwithin*{conjecture}{chapter}
\counterwithin*{justification}{chapter}
\counterwithin*{definition}{chapter}
\counterwithin*{exercise}{chapter}
\counterwithin*{example}{chapter}
\counterwithin*{counterexample}{chapter}
