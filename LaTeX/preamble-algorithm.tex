\documentclass[letterpaper]{article}

\usepackage{amsmath, amsfonts, amssymb, amstext, amscd, amsthm, bbm, booktabs, CJKutf8, caption, color, dsfont, enumerate, float, graphicx, hyperref, lipsum, listings, makecell, makeidx, mathrsfs, mathtools, marvosym, parskip, soul, stmaryrd, tikz, url, verbatim, xargs, xcolor, xfrac}
\usepackage[left=2cm,top=2cm,right=2cm,bottom=2cm,bindingoffset=0cm]{geometry}
\usepackage[group-separator={,},group-minimum-digits={3}]{siunitx}
\hypersetup{
    citecolor=black!50!red,
    colorlinks=true,
    linkcolor=black!50!red,
    urlcolor=black!25!blue
}

% define new conditionals: \if<conditional>
\newif\ifindentproofs
\newif\ifindenttheorems

% set the values of those conditionals: \<conditional>true / \<conditional>false
\indentproofstrue
\indenttheoremstrue

\allowdisplaybreaks

\newcounter{nalg}[chapter]
\renewcommand{\thenalg}{\thechapter.\arabic{nalg}}
\DeclareCaptionLabelFormat{algocaption}{\it Algorithm \thenalg}

\lstnewenvironment{algorithm}[1][]
{
    \refstepcounter{nalg}
    \captionsetup{labelformat=algocaption,labelsep=colon}
    \lstset{
        mathescape=true,
        frame=tB,
        numbers=left,
        numberstyle=\tiny,
        basicstyle=\scriptsize,
        keywordstyle=\color{black}\bfseries\em,
        keywords={,input, output, return, datatype, function, in, if, else, elif, for, foreach, while, not, begin, end, true, false, null, break, continue, let, and, or, }
        numbers=left,
        xleftmargin=.04\textwidth,
        #1
    }
}
{}

\newtheoremstyle{thm}
  {}%           Space above, empty = `usual value'
  {}%           Space below
  {\itshape}%   Body font
  {}%           Indent amount (empty = no indent, \parindent = para indent)
  {\bfseries}%  Thm head font
  {}%           Punctuation after thm head
  {}%           Space after thm head: \newline = linebreak
  {}%           Thm head spec
\newtheoremstyle{dfn}
  {}%           Space above, empty = `usual value'
  {}%           Space below
  {}%           Body font
  {}%           Indent amount (empty = no indent, \parindent = para indent)
  {\bfseries}%  Thm head font
  {}%           Punctuation after thm head
  {}%           Space after thm head: \newline = linebreak
  {}%           Thm head spec
\newtheoremstyle{rmk}
  {}%           Space above, empty = `usual value'
  {}%           Space below
  {}%           Body font
  {}%           Indent amount (empty = no indent, \parindent = para indent)
  {\bfseries}%  Thm head font
  {}%           Punctuation after thm head
  {}%           Space after thm head: \newline = linebreak
  {}%           Thm head spec

\ifindentproofs
    % begin new proof environment
    \expandafter\let\expandafter\oldproof\csname\string\proof\endcsname
    \let\oldendproof\endproof

    \renewenvironment{proof}[1][\proofname]{%
        \oldproof[#1]\quote~

    }{%
        %\endquote\oldendproof
        \endquote\qed
    }
    % end new proof environment
\else
\fi

\ifindenttheorems
    \theoremstyle{thm}% style for theorems
    \newtheorem{pretheorem}{Theorem}
    \newtheorem{prelemma}{Lemma}
    \newtheorem{preproposition}{Proposition}
    \newtheorem{precorollary}{Corollary}
    \newtheorem{preclaim}{Claim}
    \newtheorem{preconjecture}{Conjecture}
    \newtheorem{prejustification}{Justification}

    \theoremstyle{dfn}% style for definitions
    \newtheorem{preaxiom}{Axiom}
    \newtheorem{predefinition}{Definition}
    \newtheorem{preexercise}{Exercise}
    \newtheorem{preexample}{Example}
    \newtheorem{precounterexample}{Counterexample}

    \theoremstyle{rmk}% style for remarks
    \newtheorem{preidea}{Idea}
    \newtheorem*{preremark}{Remark}
    \newtheorem*{prenote}{Note}

    % theorem
    \NewDocumentEnvironment{theorem}{O{} O{}}
        {\begin{pretheorem}[#1]~#2\quote}
        {\endquote\end{pretheorem}}
    % lemma
    \NewDocumentEnvironment{lemma}{O{} O{}}
        {\begin{prelemma}[#1]~#2\quote}
        {\endquote\end{prelemma}}
    % proposition
    \NewDocumentEnvironment{proposition}{O{} O{}}
        {\begin{preproposition}[#1]~#2\quote}
        {\endquote\end{preproposition}}
    % corollary
    \NewDocumentEnvironment{corollary}{O{} O{}}
        {\begin{precorollary}[#1]~#2\quote}
        {\endquote\end{precorollary}}
    % claim
    \NewDocumentEnvironment{claim}{O{} O{}}
        {\begin{preclaim}[#1]~#2\quote}
        {\endquote\end{preclaim}}
    % conjecture
    \NewDocumentEnvironment{conjecture}{O{} O{}}
        {\begin{preconjecture}[#1]~#2\quote}
        {\endquote\end{preconjecture}}
    % justification
    \NewDocumentEnvironment{justification}{O{} O{}}
        {\begin{prejustification}[#1]~#2\quote}
        {\endquote\end{prejustification}}

    % axiom
    \NewDocumentEnvironment{axiom}{O{} O{}}
        {\begin{preaxiom}[#1]~#2\quote}
        {\endquote\end{preaxiom}}
    % definition
    \NewDocumentEnvironment{definition}{O{} O{}}
        {\begin{predefinition}[#1]~#2\quote}
        {\endquote\end{predefinition}}
    % exercise
    \NewDocumentEnvironment{exercise}{O{} O{}}
        {\begin{preexercise}[#1]~#2\quote}
        {\endquote\end{preexercise}}
    % example
    \NewDocumentEnvironment{example}{O{} O{}}
        {\begin{preexample}[#1]~#2\quote}
        {\endquote\end{preexample}}
    % counterexample
    \NewDocumentEnvironment{counterexample}{O{} O{}}
        {\begin{precounterexample}[#1]~#2\quote}
        {\endquote\end{precounterexample}}

    % idea
    \NewDocumentEnvironment{idea}{O{} O{}}
        {\begin{preidea}[#1]~#2}
        {\end{preidea}}
    % remark
    \NewDocumentEnvironment{remark}{O{} O{}}
        {\begin{preremark}[#1]~#2}
        {\end{preremark}}
    % note
    \NewDocumentEnvironment{note}{O{} O{}}
        {\begin{prenote}[#1]~#2}
        {\end{prenote}}
\else
    \theoremstyle{thm}% style for theorems
    \newtheorem{theorem}{Theorem}
    \newtheorem{lemma}{Lemma}
    \newtheorem{proposition}{Proposition}
    \newtheorem{corollary}{Corollary}
    \newtheorem{claim}{Claim}
    \newtheorem{conjecture}{Conjecture}
    \newtheorem{justification}{Justification}

    \theoremstyle{dfn}% style for definitions
    \newtheorem{axiom}{Axiom}
    \newtheorem{definition}{Definition}
    \newtheorem{exercise}{Exercise}
    \newtheorem{example}{Example}
    \newtheorem{counterexample}{Counterexample}

    \theoremstyle{rmk}% style for remarks
    \newtheorem{idea}{Idea}
    \newtheorem*{remark}{Remark}
    \newtheorem*{note}{Note}
\fi

\def\lstlistingautorefname{Algorithm}
\def\itemautorefname{Section}
\newcommand{\pretheoremautorefname}{Theorem}
\newcommand{\preaxiomautorefname}{Axiom}
\newcommand{\prelemmaautorefname}{Lemma}
\newcommand{\prepropositionautorefname}{Proposition}
\newcommand{\precorollaryautorefname}{Corollary}
\newcommand{\preclaimautorefname}{Claim}
\newcommand{\preconjectureautorefname}{Conjecture}
\newcommand{\prejustificationautorefname}{Justification}
\newcommand{\predefinitionautorefname}{Definition}
\newcommand{\preexampleautorefname}{Example}
\newcommand{\precounterexampleautorefname}{Counterexample}
\newcommand{\preideaautorefname}{Idea}
\newcommand{\axiomautorefname}{Axiom}
\newcommand{\lemmaautorefname}{Lemma}
\newcommand{\propositionautorefname}{Proposition}
\newcommand{\corollaryautorefname}{Corollary}
\newcommand{\claimautorefname}{Claim}
\newcommand{\conjectureautorefname}{Conjecture}
\newcommand{\justificationautorefname}{Justification}
\newcommand{\definitionautorefname}{Definition}
\newcommand{\exampleautorefname}{Example}
\newcommand{\counterexampleautorefname}{Counterexample}
\newcommand{\ideaautorefname}{Idea}

\newcommand*{\xline}[1][3em]{\rule[0.5ex]{#1}{0.55pt}}

\newcommand{\defniff}{\mathrel{\vcentcolon\Leftrightarrow}}
\newcommand{\iffbydefn}{$ \defniff $ }
\newcommand{\niff}{\mathrel{{\ooalign{\hidewidth$\not\phantom{"}$\hidewidth\cr$\iff$}}}}
\renewcommand{\restriction}[1]{\downharpoonright_{#1}}
\renewcommand{\qedsymbol}{\sc q.e.d.}
\renewcommand{\leq}{\leqslant}
\renewcommand{\geq}{\geqslant}
\newcommand{\conjunct}{\wedge}
\newcommand{\disjunct}{\vee}
\newcommand{\defn}{\coloneqq}
\newcommand{\xor}{~\oplus~}
\newcommand{\divides}{~|~}
\newcommand{\given}{~\middle|~}
\newcommand{\suchthat}{~\middle|~}
\newcommand{\contradiction}{~\text{\Large \Lightning}~}
\newcommand{\conj}[1]{\overline{#1}}
\newcommand{\mean}[1]{\overline{#1}}
\newcommand{\integral}[1]{\smashoperator{\int_{#1}}}
\newcommand*\diff{\mathop{}\!\mathrm{d}}
\newcommand{\E}[1]{\mathbb{E}\crochets*{#1}}
\newcommand{\Esub}[2]{\mathbb{E}_{#1}\crochets*{#2}}
\newcommand{\var}[1]{\mathrm{Var}\parens*{#1}}
\newcommand{\cov}[2]{\mathrm{Cov}\parens*{#1, #2}}
\newcommand{\der}[2]{\frac{\diff{#1}}{\diff{#2}}}
\newcommand{\dern}[3]{\frac{\diff^{#3}{#1}}{\diff{#2}^{#3}}}
\newcommand{\derm}[3]{\frac{\diff^{#3}{#1}}{\diff{#2}}}
\newcommand{\prt}[2]{\frac{\partial{#1}}{\partial{#2}}}
\newcommand{\prtn}[3]{\frac{\partial^{#3}{#1}}{\partial{#2}^{#3}}}
\newcommand{\prtm}[3]{\frac{\partial^{#3}{#1}}{\partial{#2}}}
\newcommand{\Mod}[1]{~\parens{\mathrm{mod}~#1}}

\renewcommand{\O}[1]{\mathcal{O}\parens*{#1}}
\renewcommand{\P}[1]{\mathcal{P}\parens*{#1}}
\newcommand{\C}{\mathbb{C}}
\newcommand{\N}{\mathbb{N}}
\newcommand{\Q}{\mathbb{Q}}
\newcommand{\R}{\mathbb{R}}
\newcommand{\Z}{\mathbb{Z}}

\newcommand{\cf}{\textit{c.f.}, }
\newcommand{\eg}{\textit{e.g.}, }
\newcommand{\ie}{\textit{i.e.}, }
\newcommand{\viz}{\textit{viz. }}
\newcommand{\vide}{\textit{v. }}
\newcommand{\etal}{\textit{et al. }}

\DeclareMathOperator{\lcm}{lcm}
\DeclareMathOperator*{\argmin}{arg\!\min}
\DeclareMathOperator*{\argmax}{arg\!\max}

\let\originalleft\left
\let\originalright\right
\renewcommand{\left}{\mathopen{}\mathclose\bgroup\originalleft}
\renewcommand{\right}{\aftergroup\egroup\originalright}
\newcommand{\zh}[1]{\begin{CJK}{UTF8}{gbsn}#1\end{CJK}}
\newcommand{\jp}[1]{\begin{CJK}{UTF8}{gbsn}#1\end{CJK}}

\DeclarePairedDelimiterX \inner[2]{\langle}{\rangle}{#1,#2}
\DeclarePairedDelimiter \bra{\langle}{\rvert}
\DeclarePairedDelimiter \ket{\lvert}{\rangle}
\DeclarePairedDelimiter \abs{\lvert}{\rvert}
\DeclarePairedDelimiter \cardinality{\lvert}{\rvert}
\DeclarePairedDelimiter \norm{\lVert}{\rVert}
\DeclarePairedDelimiter \set{\lbrace}{\rbrace}
\DeclarePairedDelimiter \seq{\langle}{\rangle}
\DeclarePairedDelimiter \parens{(}{)}
\DeclarePairedDelimiter \crochets{[}{]}
\DeclarePairedDelimiter \brackets{\langle}{\rangle}

\let\oldemptyset\emptyset
\let\emptyset\varnothing
\let\union\cup
\let\intersection\cap
\let\intersect\cap
